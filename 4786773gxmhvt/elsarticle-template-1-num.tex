%% This is file `elsarticle-template-1-num.tex',
%%
%% Copyright 2009 Elsevier Ltd
%%
%% This file is part of the 'Elsarticle Bundle'.
%% ---------------------------------------------
%%
%% It may be distributed under the conditions of the LaTeX Project Public
%% License, either version 1.2 of this license or (at your option) any
%% later version.  The latest version of this license is in
%%    http://www.latex-project.org/lppl.txt
%% and version 1.2 or later is part of all distributions of LaTeX
%% version 1999/12/01 or later.
%%
%% Template article for Elsevier's document class `elsarticle'
%% with numbered style bibliographic references
%%
%% $Id: elsarticle-template-1-num.tex 149 2009-10-08 05:01:15Z rishi $
%% $URL: http://lenova.river-valley.com/svn/elsbst/trunk/elsarticle-template-1-num.tex $
%%
\documentclass[preprint,12pt]{elsarticle}

%% Use the option review to obtain double line spacing
%% \documentclass[preprint,review,12pt]{elsarticle}

%% Use the options 1p,twocolumn; 3p; 3p,twocolumn; 5p; or 5p,twocolumn
%% for a journal layout:
%% \documentclass[final,1p,times]{elsarticle}
%% \documentclass[final,1p,times,twocolumn]{elsarticle}
%% \documentclass[final,3p,times]{elsarticle}
%% \documentclass[final,3p,times,twocolumn]{elsarticle}
%% \documentclass[final,5p,times]{elsarticle}
%% \documentclass[final,5p,times,twocolumn]{elsarticle}

%% The graphicx package provides the includegraphics command.
\usepackage{graphicx}
%% The amssymb package provides various useful mathematical symbols
\usepackage{amssymb}
%% The amsthm package provides extended theorem environments
%% \usepackage{amsthm}

%% The lineno packages adds line numbers. Start line numbering with
%% \begin{linenumbers}, end it with \end{linenumbers}. Or switch it on
%% for the whole article with \linenumbers after \end{frontmatter}.
\usepackage{lineno}

%% natbib.sty is loaded by default. However, natbib options can be
%% provided with \biboptions{...} command. Following options are
%% valid:

%%   round  -  round parentheses are used (default)
%%   square -  square brackets are used   [option]
%%   curly  -  curly braces are used      {option}
%%   angle  -  angle brackets are used    <option>
%%   semicolon  -  multiple citations separated by semi-colon
%%   colon  - same as semicolon, an earlier confusion
%%   comma  -  separated by comma
%%   numbers-  selects numerical citations
%%   super  -  numerical citations as superscripts
%%   sort   -  sorts multiple citations according to order in ref. list
%%   sort&compress   -  like sort, but also compresses numerical citations
%%   compress - compresses without sorting
%%
%% \biboptions{comma,round}

% \biboptions{}

\journal{Journal Name}

\begin{document}

\begin{frontmatter}

%% Title, authors and addresses

\title{Motor Represenentations of Visual Patterns}

%% use the tnoteref command within \title for footnotes;
%% use the tnotetext command for the associated footnote;
%% use the fnref command within \author or \address for footnotes;
%% use the fntext command for the associated footnote;
%% use the corref command within \author for corresponding author footnotes;
%% use the cortext command for the associated footnote;
%% use the ead command for the email address,
%% and the form \ead[url] for the home page:
%%
%% \title{Title\tnoteref{label1}}
%% \tnotetext[label1]{}
%% \author{Name\corref{cor1}\fnref{label2}}
%% \ead{email address}
%% \ead[url]{home page}
%% \fntext[label2]{}
%% \cortext[cor1]{}
%% \address{Address\fnref{label3}}
%% \fntext[label3]{}


%% use optional labels to link authors explicitly to addresses:
%% \author[label1,label2]{<author name>}
%% \address[label1]{<address>}
%% \address[label2]{<address>}

\author{}

\address{}

\begin{abstract}
%% Text of abstract
Investigate how visual patterns may be encoded by the actuator, by building a model that replicates images of English alphabets. Analyze the parameters of such a mapping, and determine those that correspond to the actuator. Vary the actuator parameters to explore the robustness of the model. 
\end{abstract}

\end{frontmatter}


%% main text
\section{Building the Model}

Find a function that maps visual inputs to actuated outputs.The requirements for the function are the following. 
\begin{itemize}
\item It should be parametrized by hidden variables.
\item It should be able to reproduce the images to a reasonable degree of accuracy. In particular it should be able to produce transitionally invariant forms of the 26 letters in the alphabet.
\item If the function is used to iteratively construct the image, the output at each iteration should ideally correspond to the strokes used to form a letter by a human.
\item The function should be distilled to neuronal network to accurately reflect computation in the brain.

\end{itemize}
	 

Simplest mechanism is to implement a fully connected auto-encoder. The number of hidden units should be greater than the number of input pixels. A locally connected layer may replace a fully connected layer so as to capture the 2d structure of the image. An iterative methodology may consist of a Recurrent Neural Network where the output at each state is the actuated output. The state at time t would be a function of the previous state and the current input. An LSTM may be used for this purpose.



\section{How Are Encoded Memories stored?}

This involves analyzing the hidden variables or the parameters of the function described above. Is there a natural way to separate these variables? Do all hidden variables correspond to encoded memories?  

The parameters corresponding to the actuator are of primary interest. Separate based on sensitivity analysis, if each parameter is modeled as a random variable. What should a motor output be invariant to. Should we just assume a Hierarchical separation, if the function is implemented in the form of a neural network?


\section{Reproducing Output through Other Actuators}

How can they be reproduced through other actuators? Change the parameters of the actuator once they have been identified, and see what the image looks like compared to the visual input. This essentially checks for robustness. 

Another option is to change the output image size, this is equivalent to changing the dynamics of the actuator and hence the actuator  itself. Compare the new learned parameters with the previously learned ones, and identify those parameters that not only parameterize the actuator representation but are also invariant to the change in the actuator.
 


%% The Appendices part is started with the command \appendix;
%% appendix sections are then done as normal sections
%% \appendix

%% \section{}
%% \label{}

%% References
%%
%% Following citation commands can be used in the body text:
%% Usage of \cite is as follows:
%%   \cite{key}          ==>>  [#]
%%   \cite[chap. 2]{key} ==>>  [#, chap. 2]
%%   \citet{key}         ==>>  Author [#]

%% References with bibTeX database:

\bibliographystyle{model1-num-names}
\bibliography{sample.bib}

%% Authors are advised to submit their bibtex database files. They are
%% requested to list a bibtex style file in the manuscript if they do
%% not want to use model1-num-names.bst.

%% References without bibTeX database:

% \begin{thebibliography}{00}

%% \bibitem must have the following form:
%%   \bibitem{key}...
%%

% \bibitem{}

% \end{thebibliography}


\end{document}

%%
%% End of file `elsarticle-template-1-num.tex'.